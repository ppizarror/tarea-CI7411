Utilizando el mismo modelo de capas en el programa DEEPSOIL se calculó la función de trasferencia de fourier para los registros sísmicos en superficie. En la Figura \ref{p3gemp} se ilustra el resultado obtenido con el programa, junto con la curva promedio y la de desviación estándar. \\

Al comparar el resultado empírico con el medido (parte 3.c de la tarea) se puede observar que, nuevamente, para frecuencias bajas tanto el modelo empírico como el medido presenta un comportamiento similar con respuestas de amplificación en torno a frecuencias similares, $1Hz$; sin embargo, a mayor energía, existe una diferencia notable en cuanto al comportamiento.

\insertimage[\label{p3hemp}]{p3h-emp}{width=12cm}{Función de transferencia empírica calculada a partir del método lineal equivalente para el caso de los registros sísmicos en la superficie.}

\insertimage[\label{p3hcomp}]{p3h-comp}{width=12cm}{Comparación de la función de transferencia entre el caso empírico y medido.}

\newpage
Mediante DEEPSOIL se obtuvo además los espectros de aceleración para distintos períodos, en la Figura \ref{p3h-sa} se ilustra tanto el promedio como la desviación estándar.

\insertimage[\label{p3h-sa}]{p3h-sa}{width=12cm}{Aceleración espectral obtenido mediante el método lineal equivalente para los registros en superficie.}