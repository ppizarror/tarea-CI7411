Para el desarrollo de esta pregunta se obtuvo la FFT del registro en la componente norte-sur y este-oeste para los registros tanto en roca como en superficie. Al igual que el caso anterior se consideró el promedio geométrico de los registros horizontales. \\

Para eliminar la gran cantidad de ruido en la FFT, se utilizó una ventana de Tuckey con un ancho de 5\%, se corrigó por línea base, y se aplicó un filtro (Mean Filter) para suavizar la transformada. \\

Con los $FFT_H$ de cada registro roca/superficie el cálculo de la función de transferencia empírica del sitio con respecto a la base comprende:

\insertequation{FT = \frac{FFT_{H}^{\ suelo}}{FFT_{H}^{\ roca}}}

En la Figura \ref{p3c} se ilustra el resultado obtenido. Como se puede observar existen sucesivos peaks para la amplificación en superficie. Para frecuencias mas grandes se tiene una menor amplificación.

\insertimage[\label{p3c}]{p3c}{width=12cm}{Función de transferencia empírica del sitio con respecto a la base del depósito.}