\newpage
\section{Pregunta 3}

En el contexto de un estudio de caracterización sísmica de la ciudad de Llolleo, se instrumenta con acelerógrafos un sondaje que llega hasta la roca basal y se realizan una serie de mediciones in-situ. Dentro de los datos disponibles se tiene:

\begin{itemizebf}
	\item Down-Hole con mediciones de Vs cada 5 m hasta la roca basal, SPT y estratigrafía completa del sitio en artículo de Verdugo (2009).
	\item Mediciones de vibraciones ambientales en superficie.
	\item Registros sísmicos de aceleración medidos en Llolleo a nivel de roca basal y superficie.
	\item Registros sísmicos de aceleración del Terremoto del Maule de 2010, medidos en distintas estaciones sobre roca.
\end{itemizebf}

\subsection{Calcule la razón espectral H/V del sitio, utilizando las mediciones de vibraciones ambientales procesadas con el software Geopsy}
A partir de las mediciones de vibraciones ambientales en superficie de la ciudad de Llolleo (tanto en ciudad como en ubicaciones aledañas) se calculó la razón espectral H/V tanto de cada medición como el promedio del conjunto. \\

Este cálculo se hizo a partir del programa Geopsy, el cual carga los registros, aplica filtros, selecciona ventanas de medición (de 30 segundos, Tuckey window con ancho de 5\%), suaviza la FFT (Konno \& Ohmachi) y calcula el H/V \footnote{H/V, según lo estudiado por Nakamura, establece la razón espectral entre las componentes horizonales y verticales de un registro sísmico de un suelo. Cuando H/V es superior a 2 para un cierto rango de frecuencias se entiende que hay amplificación sísmica, dado un fuerte contraste de impedancias. En roca, por otra parte, H/V debe oscilar entre 1 y 2.}.

\insertimage{p3a-selventana}{width=7cm}{Selección automática de las ventanas de medición, Geopsy.}

En la Figura \ref{p3a-hv-city} se puede observar el H/V para el registro de la ciudad de Llolleo, en la Figura \ref{p3a-hv-exp} se ilustra los resultados de H/V para las cercanías de la ciudad de Llolleo. Por último en la Figura \ref{p3a-hv-prom} se ilustra el promedio H/V de cada set de registros.

\begin{images}[\label{p3a-hv-city}]{H/V para registros vibraciones en la ciudad de Llolleo.}
	\addimage{p3a-hv-city1}{width=7cm}{H/V registro vibración ambiental city1.001.}
	\addimage{p3a-hv-city2}{width=7cm}{H/V registro vibración ambiental city1.002.}
	\addimage{p3a-hv-city3}{width=7cm}{H/V registro vibración ambiental city1.003.}
	\addimage{p3a-hv-city4}{width=7cm}{H/V registro vibración ambiental city1.004.}
\end{images}

\begin{images}[\label{p3a-hv-exp}]{H/V para registros vibraciones en las cercanías de la ciudad de Llolleo.}
	\addimage{p3a-hv-exp1}{width=7cm}{H/V registro vibración ambiental exp1\_dgf1.}
	\addimage{p3a-hv-exp2}{width=7cm}{H/V registro vibración ambiental exp2\_dgf1.}
	\addimage{p3a-hv-exp3}{width=7cm}{H/V registro vibración ambiental exp3\_dgf1.}
	\addimage{p3a-hv-exp4}{width=7cm}{H/V registro vibración ambiental exp4\_dgf1.}
\end{images}

\begin{images}[\label{p3a-hv-prom}]{H/V promedios de registros.}
	\addimage{p3a-hv-cityprom}{width=7cm}{Promedio H/V registro vibración ambiental en ciudad de Llolleo.}
	\addimage{p3a-hv-expprom}{width=7cm}{Promedio H/V registro vibración ambiental cercanías ciudad de Llolleo.}
\end{images}

Como se puede observar en las figuras anteriores el H/V en la ciudad de Llolleo indica una clara amplificación en torno a la frecuencia de $2Hz$, con lo cual da a entender un claro contraste de impedancias entre roca basal y superficie. \\

Por otro lado, como se puede apreciar en la Figura \ref{p3a-hv-prom}.b el promedio de H/V en las cercanías de la ciudad de Llolleo también indica un claro contraste de impedancias, sin embargo presenta una frecuencia ligeramente superior, en torno a los $2.2Hz$. Esto ocurre principalmente porque los suelos son distintos entre ambos sitios de medición.

\newpage
\subsection{Calcule la razón espectral H/V del sitio, utilizando los espectros de respuesta de los registros sísmicos en superficie. Compare el promedio de estos resultados con el promedio de los resultados de la parte a}
Utilizando los registros sísmicos en superficie se calculó la razón espectral H/V del sitio utilizando el cuociente de la transformada de Fourier de la componente horizontal sobre la vertical. A modo de poder absorber ambos sentidos (E-W, N-S) se utilizó la raíz cuadrada de la suma de los cuadrados.

\insertequationanum{HV = \frac{\sqrt{FFT_{NS}^2 + FFT_{EW}^2}}{FFT_{Z}}}

Un aspecto importante del cálculo es que, para eliminar la gran cantidad de ruido en la FFT, se utilizó una ventana de Tuckey con un ancho de 5\%, se corrigó por línea base, y se aplicó un filtro (Mean Filter) para suavizar la transformada, con un ancho de 300 puntos \footnote{Valor de la ventana de filtro obtenido de forma netamente empírica.}. \\

En la Figura \ref{p3b-hv} se ilustra el resultado obtenido con Matlab. La línea gruesa corresponde al promedio, las líneas segmentadas indican la desviación estándar.

\insertimage[\label{p3b-hv}]{p3b-hv}{width=12cm}{H/V Registro superficie.}

Es posible observar que el período predominante de H/V es cercano a los 2 $Hz$, lo cual es consistente con lo obtenido en la parte (a), dado que el suelo es el mismo. Esto es bastante interesante dado que se obtienen períodos similares con dos metodologías distintas (vibraciones ambientales y registros sísmicos). 

\subsection{Calcule la función de transferencia empírica del sitio con respecto a la base del depósito, considerando todos los registros sísmicos disponibles}
Para el desarrollo de esta pregunta se obtuvo la FFT del registro en la componente norte-sur y este-oeste para los registros tanto en roca como en superficie. Al igual que el caso anterior se consideró el promedio geométrico de los registros horizontales. \\

Para eliminar la gran cantidad de ruido en la FFT, se utilizó una ventana de Tuckey con un ancho de 5\%, se corrigó por línea base, y se aplicó un filtro (Mean Filter) para suavizar la transformada. \\

Con los $FFT_H$ de cada registro roca/superficie el cálculo de la función de transferencia empírica del sitio con respecto a la base comprende:

\insertequation{FT = \frac{FFT_{H}^{\ suelo}}{FFT_{H}^{\ roca}}}

En la Figura \ref{p3c} se ilustra el resultado obtenido. Como se puede observar existen sucesivos peaks para la amplificación en superficie. Para frecuencias mas grandes se tiene una menor amplificación.

\insertimage[\label{p3c}]{p3c}{width=12cm}{Función de transferencia empírica del sitio con respecto a la base del depósito.}

\subsection{Calcule las funciones de transferencia (FTsb y FTsa) calculadas a partir de la teoría unidimensional de ondas de corte para un medio visco-elástico multicapa vista en clases. Considere estas soluciones como los resultados analíticos del problema}
Para resolver esta parte de la pregunta se tuvo que modelar el perfil de suelo utilizando el código programado en la pregunta 2 para un medio visco-elástico multicapa. Para ello se obtuvieron los datos a partir de la estatigrafía presentada por el artículo de Ramón Verdugo \cite{ref4}, ilustrada en la Figura \ref{verdugolayer}.

\insertimage[\label{verdugolayer}]{layer-verdugo}{height=15cm}{Estatigrafía del suelo de Llolleo.}

\newpage
Si bien se recomienda utilizar un sistema de cuatro capas para modelar el perfil de suelos se decidió finalmente utilizar un total de 8 capas, ya que esto permite obtener una solución más parecida a la realidad sin mayor esfuerzo. En la Tabla \ref{tab:p4d} se detallan las capas utilizadas.

\begin{table}[H]
	\centering
	\caption{Sistema de capas utilizado para la modelación.}
	\begin{tabular}{|P{5em}|P{6em}|c|c|}
		\hline
		\textbf{N° Layer} & \textbf{Espesor [m]} & \multicolumn{1}{p{5.39em}|}{\textbf{Vs [m/s]}} & \multicolumn{1}{p{5.39em}|}{\textbf{$\rho$ [KN/m3]}} \bigstrut\\
		\hline
		\multicolumn{1}{|c|}{1} & \multicolumn{1}{c|}{6} & 180   & 18 \bigstrut\\
		\hline
		\multicolumn{1}{|c|}{2} & \multicolumn{1}{c|}{5} & 200   & 15 \bigstrut\\
		\hline
		\multicolumn{1}{|c|}{3} & \multicolumn{1}{c|}{2} & 250   & 20 \bigstrut\\
		\hline
		\multicolumn{1}{|c|}{4} & \multicolumn{1}{c|}{8} & 200   & 16 \bigstrut\\
		\hline
		\multicolumn{1}{|c|}{5} & \multicolumn{1}{c|}{5} & 720   & 21 \bigstrut\\
		\hline
		\multicolumn{1}{|c|}{6} & \multicolumn{1}{c|}{8} & 250   & 17 \bigstrut\\
		\hline
		\multicolumn{1}{|c|}{7} & \multicolumn{1}{c|}{11} & 250   & 17 \bigstrut\\
		\hline
		\multicolumn{1}{|c|}{8} & \multicolumn{1}{c|}{16} & 720   & 19 \bigstrut\\
		\hline
		Roca  & -     & 1800  & 25 \bigstrut\\
		\hline
	\end{tabular}
	\label{tab:p4d}
\end{table}

Es importante mencionar que para entregar un peso unitario para cada estrato, se recurrió a correlaciones con el N-SPT, eligiendo valores entre los rangos presentados a continuación:

\insertimageboxed{nspt1}{width=11cm}{0.5}{Correlación SPT - Peso unitario suelo (Suelos granulares).}
\insertimageboxed{nspt2}{width=11cm}{0.5}{Correlación SPT - Peso unitario suelo (Suelos cohesivos).}

Con lo anterior y utilizando la derivación de la función programada \texttt{u\_velt} se obtuvo los parámetros $E_i$ y $F_i$ del suelo para cada uno de los $n$ estratos. En este sentido:

\insertequationanum{FT_{sb} = \frac{2E_1}{E_{n+1} + F_{n+1}} \quad \quad FT_{sa} = \frac{E_1}{E_{n+1}}} \\

Evaluando con el sistema de capas:

\begin{images}[\label{p3dimg}]{Funciones de transferencia para el medio visco-elástico del perfil de Llolleo.}
	\addimage{p3d-ftsb}{width=7cm}{$FT_{sb}$}
	\addimage{p3d-ftsa}{width=7cm}{$FT_{sa}$}
\end{images}

Como era esperado se puede observar en la Figura \ref{p3dimg} que el factor de transferencia posee mayores amplitudes dado el mayor contraste de impedancias.

\subsection{Evalúe el movimiento del depósito utilizando el código desarrollado en la pregunta 2. Simplifique el perfil para que coincida con cuatro capas}
Utilizando el mismo sistema de estratos y propiedades de la parte d se obtuvo el siguiente perfil de desplazamiento $u(z,t)$ para un sistema visco-elástico multicapa.

\insertimage{p3e}{width=8cm}{Movimiento del depósito utilizando el código \texttt{u\_velt}.}

\subsection{Calcule la función de transferencia calculada a partir del método lineal equivalente (programa Deepsoil o equivalente) utilizando los registros sísmicos en la base. Compare con los resultados analíticos y empíricos}
En el programa DEEPSOIL se modeló el perfil de suelos detallado en la Tabla \ref{tab:p4d} utilizando una curva de degradación del material distinta para cada capa. Estas curvas fueron obtenidas a partir de la literatura, proponiendo los siguientes modelos para cada uno de los estratos:

\begin{table}[H]
	\centering
	\caption{Modelos utilizados para cada estrato.}
	\itemresize{0.8}{
		\begin{tabular}{|c|P{8em}|P{11.5em}|P{11em}|}
			\hline
			\multicolumn{1}{|P{3em}|}{\textbf{N° Layer}} & \textbf{Suelo} & \textbf{Fuente curva de degradación de rigidez} & \textbf{Plasticidad promedio estrato fino} \bigstrut\\
			\hline
			1     & Arena limosa pobremente graduada & Seed \& Idriss. 1991 (Promedio) & - \bigstrut\\
			\hline
			2     & Arcilla y limo de baja plasticidad & Vucetic \& Dobry. 1991 & \multicolumn{1}{c|}{15} \bigstrut\\
			\hline
			3     & Arena limosa pobremente graduada & Seed \& Idriss. 1991 (Promedio) & - \bigstrut\\
			\hline
			4     & Arcilla y limo de baja plasticidad & Vucetic \& Dobry. 1991 & \multicolumn{1}{c|}{18} \bigstrut\\
			\hline
			5     & Grava y arena pobremente graduada & Rollins. 1998 & - \bigstrut\\
			\hline
			6     & Arcilla y limo de baja plasticidad & Vucetic \& Dobry. 1991 & \multicolumn{1}{c|}{15} \bigstrut\\
			\hline
			7     & Arcilla de alta plasticidad & -     & \multicolumn{1}{c|}{20} \bigstrut\\
			\hline
			8     & Arena limosa con arcilla pobremente graduada & -     & - \bigstrut\\
			\hline
		\end{tabular}
	}
\end{table}

En la Figura \ref{curvag} se ilustra la curva de degradación utilizada para la Arena y Arcilla (estratos 7 y 8 respectivamente).

\insertimage[\label{curvag}]{curvag}{width=7cm}{Curva de degradación utilizada para la Arena.}

\newp El perfil modelado en DEEPSOIL se ilustra en la Figura \ref{modelodpsoil}.

\insertimage[\label{modelodpsoil}]{modelodeepsoil}{width=3.5cm}{Perfil de los estratos de Llolleo en DEEPSOIL: Total profile Depth: $61m$, Natural Freq. of profile: $1.18Hz$, Natural period: $0.85s$.}

Utilizando los registros sísmicos en la roca basal se corrió el modelo obteniendo la función de transferencia de fourier del programa para cada registro. Los resultados de todos los registros fueron graficados con matlab, en donde se calculó el promedio y la desviación estándar (Figura \ref{ftemproca}).

\insertimage[\label{ftemproca}]{p3f-emp}{width=11cm}{Función de transferencia empírica calculada a partir del método lineal equivalente para el caso de los registros sísmicos en la base.}

Al contrastar la función de transferencia empírica (Figura \ref{ftemproca}) con la analítica (Figura \ref{p3dimg}.a) se puede observar que tanto el modelo analizado con la teoría viscoelástica (código pregunta 2) y con el perfil obtenido a partir del modelo lineal equivalente (DEEPSOIL) presentan amplificaciones en frecuencias similares.

\insertimage{p3f-comp}{width=11cm}{Comparación resultados analíticos y empíricos para el registro sísmico en roca basal.}

\subsection{Calcule la función de transferencia calculada a partir del método lineal equivalente (programa Deepsoil) utilizando los registros en roca del Terremoto del Maule de 2010 (recuerde que el input sísmico se considera en un afloramiento rocoso). Compare con los resultados analíticos y empíricos}
Utilizando el mismo modelo de capas en el programa DEEPSOIL se calculó la función de trasferencia de fourier para los registros sísmicos en afloramiento rocoso, correspondientes al terremoto del Maule del 2010. En la Figura \ref{p3gemp} se ilustra el resultado obtenido con el programa, junto con la curva promedio y la de desviación estándar.

\insertimage[\label{p3gemp}]{p3g-emp}{width=12cm}{Función de transferencia empírica calculada a partir del método lineal equivalente pra el caso de los registros sísmicos en la base.}

Es claro ver que el sismo en afloramiento rocoso presenta una amplificación para frecuencias bajas (propias de un sistema rígido), para frecuencias mayores no existe la energía suficiente como para hacer excitar el medio. \\

Al comparar la función de transferencia empírica con la analítica (obtenida del código viscoelástico \texttt{u\_velt}) se obtuvo el gráfico de la Figura \ref{p3gcomp}. Al realizar un análisis sobre los resultados obtenidos es claro ver que para frecuencias bajas tanto el caso analítico como empírico amplifican para frecuencias bajas en torno a $1Hz$ (entre $0.88Hz$ y $1.28Hz$), sin embargo para frecuencias mayores el sismo no fue capaz de otorgar mayor energía al sistema, por tanto se obtuvieron menores desplazamiento en superficie, disminuyendo así el factor de transferencia.

\insertimage[\label{p3gcomp}]{p3g-comp}{width=12cm}{Comparación resultados analíticos y empíricos para el registro sísmico en afloramiento rocoso.}

\subsection{Determine con el método lineal equivalente las aceleraciones y los espectros de respuesta en superficie, considerando ambas componentes horizontales. Compare estos resultados con los registros de aceleración y los espectros medidos. Además, compare los espectros de respuesta con los propuestos por la norma NCh433 para el sitio asociado al Vs30 medido}
Utilizando el mismo modelo de capas en el programa DEEPSOIL se calculó la función de trasferencia de fourier para los registros sísmicos en superficie. En la Figura \ref{p3gemp} se ilustra el resultado obtenido con el programa, junto con la curva promedio y la de desviación estándar. \\

Al comparar el resultado empírico con el medido (parte 3.c de la tarea) se puede observar que, nuevamente, para frecuencias bajas tanto el modelo empírico como el medido presenta un comportamiento similar con respuestas de amplificación en torno a frecuencias similares, $1Hz$; sin embargo, a mayor energía, existe una diferencia notable en cuanto al comportamiento.

\insertimage[\label{p3hemp}]{p3h-emp}{width=12cm}{Función de transferencia empírica calculada a partir del método lineal equivalente para el caso de los registros sísmicos en la superficie.}

\insertimage[\label{p3hcomp}]{p3h-comp}{width=12cm}{Comparación de la función de transferencia entre el caso empírico y medido.}

\newpage
Mediante DEEPSOIL se obtuvo además los espectros de aceleración para distintos períodos, en la Figura \ref{p3h-sa} se ilustra tanto el promedio como la desviación estándar.

\insertimage[\label{p3h-sa}]{p3h-sa}{width=12cm}{Aceleración espectral obtenido mediante el método lineal equivalente para los registros en superficie.}