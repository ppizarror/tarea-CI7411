\newpage
\section{Síntesis de resultados}

A modo de síntesis de los comentarios expuestos en las secciones anteriores del presente informe es posible concluir, en una primera instancia, que el método de Nakamura para obtener la frecuencia de amplificación es consistente, dado que, tanto para el caso de vibración ambiental (Figura \ref{p3a-hv-prom}) como la razón espectral mediante los registros sísmicos en superficie (Figura \ref{p3b-hv}) se obtuvo un período predominante similar ($2H_z$).

\newp Sin embargo, mediante esta técnica no es posible saber en cuánto amplifica, dado que el valor peak de H/V no es igual en ambos casos, por tanto, si se necesita un resultado más detallado es necesario el uso de una metodología distinta. Lo bueno de esta metodología es que es económica, ya que no se necesita hacer perforaciones o sondajes.

\newp En cuanto a la función de transferencia mediante registros sísmicos se obtuvieron resultados muy parecidos para el caso analítico y empírico con el registro en roca basal (Figura \ref{rftroca}), obteniendo factores de amplificación idénticos en frecuencias bajas y parecidos en frecuencias mas altas.

\newp En cuanto a los registros en afloramiento rocoso se obtuvo un error mayor en la función de transferencia, tanto en frecuencias como en la amplitud (Figura \ref{p3gcomp}), por tanto es posible que el sismo no haya sido capaz de entregar energía suficiente al sistema del suelo como para lograr desplazamientos mayores en superficie, ello explicaría el por qué la amplitud del factor de transferencia es tan bajo para frecuencias mayores, otro origen del error puede ser efectos no lineales del suelo dado que ese rango de frecuencias es característico de terremotos de alta energía, por tanto la rigidez del suelo puede cambiar excesivamente para mayores frecuencias, cambiando el comportamiento.

\newp Un fenómeno parecido ocurre al contrastar la función de transferencia empírica con el registro en superficie (Figura \ref{p3hcomp}), para frecuencias más bajas se obtuvo una frecuencia similar de amplificación (cerca del $1Hz$) sin embargo para frecuencias mayores ocurren desviaciones tanto en frecuencia como amplitud.

\newp Lo anterior permite concluir que si se requiere estudiar la función de transferencia de un suelo es recomendable utilizar un registro sísmico en la roca basa, dado que esto disminuye la fuente de error tanto para frecuencias bajas y altas. Sin embargo realizar estudios a este nivel de profundidad (En el caso de Llolleo, 61 metros) es extremadamente costoso.