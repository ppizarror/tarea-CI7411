\section{Pregunta 1}

\textbf{Implemente las ecuaciones de propagación unidimensional de onda de corte para un medio elástico en condición permanente y grafique el desplazamiento de partícula en profundidad en forma animada. Siga el ejemplo del software Quake visto en clases (disponible en material docente de u-cursos).} \\

\subsectionanum{Teoría}
Para estudiar la propagación unidimensional de onda de corte para un medio elástico se realiza equilibrio de fuerzas en un elemento infinitesimal sometido a corte dentro del medio, tal como se observa en la Figura \ref{p1corte}, en donde $\rho$ es la densidad del medio, $G$ el módulo de corte y $H$ la altura del medio elástico.

\insertimage[\label{p1corte}]{p1corte}{width=8cm}{Elemento infinitesimal sometido a corte.}

Al realizar equilibrio dinámico de fuerzas en el eje $u$ se tiene que:

\insertequation{\underequal{\bigg(\cancel{\tau }+ \fracpartial{\tau}{z}\cancel{\dd z}\bigg)\cancel{\dd x \dd y} - \cancel{\tau \dd x \dd y}}{\textit{Eq. fuerzas}} = \underequal{\rho \cancel{\dd x \dd y} \cancel{\dd z}}{\textit{Masa}} \cdot \underequal{\fracdpartial{u}{t}}{\textit{Aceleración}}}

\insertequation[\label{p1tauz}]{\fracpartial{\tau}{z} = \rho \cdot \fracdpartial{u}{t}}

Para un material elástico se tiene que el corte es directamente proporcional a la deformación angular, con constante de proporción el módulo de corte o cizalle $G$:

\insertequation[\label{leycorte}]{\tau = G \cdot \gamma = G \cdot \fracpartial{u}{z}}

Al utilizar \eqref{leycorte} en \eqref{p1tauz} se obtiene finalmente:

\insertequation[\label{p1gonda}]{G\cdot\fracdpartial{u}{z} = \rho\cdot\fracdpartial{u}{t}}

La forma de la ecuación anterior \eqref{p1gonda} es similar a la ecuación de onda $\fracdpartial{\xi}{t} = \nu^2 \cdot \fracdpartial{\xi}{x}$, conocida también como la ecuación de D'Alembert, en donde describe una onda de velocidad $\nu$ relacionando tanto espacio como tiempo. \\

Considerando luego que $V_s^2=\frac{\rho}{G}$ corresponde a la velocidad de la onda de corte se puede utilizar la solución característica de la ecuación de onda como solución del problema de propagación unidimensional de onda de corte para un medio elástico:

\insertequation{u(t,z)=A \cdot e^{i\big(\omega t + k z\big)}}

Cuyas derivadas corresponden a:

\insertequation[\label{p1udervs}]{\begin{array}{ccc}
	\fracpartial{u}{t} = Ai\omega \cdot e^{i\omega t} \cdot e^{ikz} & & \fracdpartial{u}{t} = - A \omega^2  \cdot e^{i\omega t} \cdot e^{ikz} \\
	&& \\
	\fracpartial{u}{z} = Aik \cdot e^{i\omega t} \cdot e^{ikz} & &  \fracdpartial{u}{z} = -A k^2 \cdot e^{i\omega t} \cdot e^{ikz}
	\end{array}
}\\

Reemplazando \eqref{p1udervs} en \eqref{p1gonda} se obtiene la siguiente relación:

\insertequationanum{-G A k^2 \cdot e^{i\omega t} \cdot e^{ikz} = -\rho A \omega^2  \cdot e^{i\omega t} \cdot e^{ikz}}
\insertequation[\label{Grhovs}]{\frac{G}{\rho} = \bigg(\frac{\omega}{k}\bigg)^2 = {V_s}^2}

En donde $k$ corresponde al número de onda, $k = \frac{2\pi}{\lambda}$, $\lambda$ la longitud de la onda. Finalmente, suponiendo que $-z$ también es solución de la ecuación se obtiene la solución final:

\insertequation{\boxed{u(z,t)=A\cdot e^{i\omega \big(t+\frac{z}{V_s}\big)} + B\cdot e^{i\omega \big(t-\frac{z}{V_s}\big)}}}

Para resolver el problema de un estrato sobre roca es necesario identificar las condiciones de borde del problema:

\begin{itemize}
	\item \textbf{Condición natural de borde en superficie, no transmite corte} \\
	En este caso se tiene que $\tau(z=0)=0$, por lo tanto $\fracpartial{u}{z} \big|_{z=0}=0$:
	
	\insertequationanum{\fracpartial{u}{z} = A \cdot \frac{i\omega}{V_s} \cdot e^{i\omega t}\cdot e^{i\omega \frac{z}{V_s}}-B\cdot \frac{i\omega}{V_s} \cdot e^{i\omega t}\cdot e^{i\omega \frac{z}{V_s}} = 0 \longmapsto A=B}
	
	Así la solución se reduce a $u(z,t)=A\cdot e^{i\omega t} \bigg(e^{i\omega \frac{z}{V_s}}+e^{i\omega \frac{-z}{V_s}}\bigg) = 2A e^{i\omega t}\cos\big(\frac{\omega z}{V_s}\big)$
	
	\item \textbf{En el basamiento se tiene igual desplazamiento que en roca} \\
	
	Para este caso se tiene que $\norm{u(z,t=H)}=a_b = 2A\cdot\cancelto{1}{\norm{e^{i\omega t}}}\cos\big(\frac{\omega H}{V_s}\big) = 2A \cos\big(\frac{\omega z}{V_s}\big)$, en donde $a_b$ corresponde a la amplitud basal del movimiento en la base $a_b$, así:
	
	\insertequation{A = \frac{a_b}{2\cos\big(\frac{\omega H}{V_s}\big)}}
\end{itemize}

Por tanto, con ambas condiciones de borde se obtiene la solución para el problema:

\insertequation[\label{p1solfinal}]{\boxed{u(z,t) = \frac{a_b}{\cos\big(\frac{\omega H}{V_s}\big)}e^{i\omega t}\cos \big(\frac{\omega z}{V_s}\big)}}

\subsectionanum{Implementación en Matlab}

Con el fin de poder implementar y graficar el desplazamiento de partícula en profundidad en forma animada se desarrolló la ecuación \eqref{p1solfinal} en matlab (ver archivo \texttt{u\_elt.m}). En dicha función se pide como parámetro $T$ período, $H$ la altura del estrato, $V_s$ velocidad de corte y $a_b$ amplitud basal. \\

Dicha función retorna otra función que pide tanto $z$ como $t$ para entregar el desplazamiento, esto se desarrolló así para hacer el análogo a $u(z,t)$, en otras palabras:

\insertequationanum{u\_elt(V_s,H,a_b,T) \mapsto u(z,t)}

\begin{sourcecodep}{matlab}{firstnumber=27}{Código más importante de \texttt{u\_elt.m}.}
%% Calcula la frecuencia
w = 2 * pi / T;
cosval = cos(w*H/Vs);
if cosval == 0
	error('El periodo de la onda genera resonancia');
elseif abs(cosval) < 1e-15
	warning('El periodo de la onda está cerca de la resonancia, posible inestabilidad numérica');
end

%% Retorna la funcion de desplazamiento
u = @(z, t) (ab / cosval) * exp(1i*w*t) * cos(w*z/Vs);
\end{sourcecodep}

\newp Al tener $u(z,t)$ la generación del gráfico animado es trivial, ya que se mantiene un ciclo \textit{for} en donde se calcula la posición para un arreglo de $z$ en un determinado $t_i$, al finalizar la instancia del ciclo se actualiza el tiempo $t_{i+1} = t_{i} + \dd t$. Para mantener la velocidad de la animación se hace uso de la instrucción \textit{pause(t)} el cual permite pausar la ejecución del programa en un determinado tiempo $t$. La función que se encarga de graficar la animación corresponde a \texttt{quake\_elt}.

\subsectionanum{Validación de la metodología}

A modo de testear que el código sea consistente se crearon dos tests, uno tiene por objetivo contrastar el resultado obtenido de forma analítica, y en otro caso se prueba el código en el estado de resonancia.

\subsubsectionanum{Test analítico}

Sea un estrato de $H$=20 metros de profundidad, con una velocidad de corte de $V_s$=150 metros por segundo. En este caso, para cualquier valor de amplitud basal $a_b$, sea $a_b$=10 metros, evaluado en un período de $T=$1.8 segundos, se tiene:

\insertequationanum{\omega = \frac{2\pi}{T} = 3.4907\ Hz \quad \quad u(z,t)=\frac{10}{\cos\big(\frac{3.4907 \cdot 20}{150}\big)}e^{i3.4907 t}\cos \big(\frac{3.4907 z}{150}\big)}
\insertequationanum{u(z,t)=11.1907e^{i3.4907 t}\cos \big(0.0233\cdot z\big)}

Algunos puntos de prueba:

\begin{table}[H]
	\centering
	\caption{Casos de prueba función y resultados analíticos}
	\begin{tabular}{ccc}
		\hline
		u (m) & t (s) & u(z,t) \bigstrut\\
		\hline
		0     & 0     & 11.1903 \bigstrut[t]\\
		20    & 0     & 10.0000 \\
		10    & 5     & 1.8908 \\
	\end{tabular}
\end{table}

Por otra parte, teóricamente el factor de amplificación corresponde a $FA=\frac{1}{\cos\bigg(\frac{\omega H}{V_s}\bigg)} = 1.1191$.

El script programado en \texttt{test\_p1\_analitico.m} retornó los siguientes gráficos:

\begin{images}{Resultado caso analítico.}
	\addimage{p1fa}{width=7cm}{Gráfico factor de amplificación.}
	\addimage{p1uzt}{width=7cm}{Quake.}
\end{images}

\newpage
\subsubsectionanum{Test resonancia}

En este test se busca llegar a la resonancia. Para ello se utilizó el mismo suelo, pero evaluado en $T=\frac{4H}{V_s}$, valor del período para el cual el coseno del denominador de la expresión se indefine, logrando teóricamente un factor de amplificación infinito.

\insertimage{p1resonancia}{width=7cm}{Evaluación del caso en resonancia.}

Tal como se puede observar en la figura anterior, se obtuvo un FA muy alto, numéricamente inestable. Esto indica claramente que la solución es correcta, ya que amplifica en el período teóricamente resonante.